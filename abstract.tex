% abstract.tex
% squinch space a bit if needed
% \setlength{\parindent}{1in}
% \setlength{\parskip}{0.5ex}
\section*{Problem}

\subsection*{Problem Definition}
    The problem of finding the maximum flow in a given graph is solved using
    preflow concept of Karzanov. By incorporating the dynamic tree data
    structure~\cite{Sleator:1983:DSD:61337.61338} of Sleator and Tarjan 
    the new approach acheives a running time of $\mathcal{O}(nm\log{(n^2/m)})$ on an 
    \textit{n-vertex}, \textit{m-edge} graph. \newline
    
    A minimum cut is a cut of minimum capacity. The max-flow, min-cut theorem 
    of Ford and Fulkerson states that the value of a maximum flow is equal 
    to the capacity of a minimum cut. \newline

    Graph G = (V, E) is a directed graph with vertex set V and edge set E.
    Size of V is denoted by \textit{n} and size of E by \textit{m}. G is a
    network if it has two distinct distinguished vertices, a \textit{source s}
    and a \textit{sink t}, and a positive capacity \textit{c(v,w)} on each
    directed edge (v,w). A flow \textit{f} on G is a real-valued function on
    vertex pairs satisfying. The value of a flow \textit{f} is the net flow
    into the sink,
    \begin{equation}
    |f| = \sum\nolimits_{v \in V} f(v,t)
    \end{equation}
    \newline

    A maximum flow is a flow of maximum value. A \textit{cut} S, $\bar S$ is a
    partition of the vertex set $( S \cup \bar S\ = \ V, S \cap \bar S \ = \
    0)$ with $ s \in S$ and $t \in \bar S$. The capacity of the cut is
    \begin{equation}
    c(S, \bar S) = \sum\nolimits_{v \in S, w \in \bar S} c(v,w)
    \end{equation}
    \newline
    
    The flow across the cut is
    \begin{equation}
    f(S, \bar S) = \sum\nolimits_{v \in S, w \in \bar S} f(v,w) = |f|
    \end{equation}
    \newline
    

\subsection*{Main Result}
    The new approach makes the algorithm as fast as any known method for any
    graph density and faster on graphs of moderate destiny. And the algorithm
    is efficient in distributed and parallel implementations. The parallel
    implementation running in $\mathcal{O} (n^2\ \log{n})$ time and uses only
    $\mathcal{O}(m)$ space with a time bound $\mathcal{O}(n^3)$.

\subsection*{Importance of Result}
    \bf{TODO:Explain why this result is important (provide some background with
    references)}

\subsection*{Impact}
    \bf{TODO:Explain what impact (if any) this result has had (or might have)}
%

